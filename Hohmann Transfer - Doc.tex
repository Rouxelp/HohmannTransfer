\documentclass[a4paper,12pt]{article}
\usepackage{amsmath}
\usepackage{amssymb}
\usepackage{graphicx}
\usepackage{hyperref}

\title{Hohmann Transfer Calculation Process}
\author{}
\date{}

\begin{document}

\maketitle

\section*{Introduction}
A Hohmann transfer is a highly efficient two-impulse orbital maneuver used to transfer a spacecraft between two circular orbits of different radii around the same central body. This document explains the process and calculations implemented in the \texttt{func\_calculate\_transfer} function.

\section*{Key Parameters}
Let:
\begin{itemize}
    \item $\mu$ be the standard gravitational parameter of the central body (in $\text{km}^3/\text{s}^2$).
    \item $r_1$ be the radius of the initial orbit (perigee), defined as:
    \begin{equation}
        r_1 = R + h_1,
    \end{equation}
    where $R$ is the radius of the central body and $h_1$ is the altitude of the initial orbit (in $\text{km}$).
    \item $r_2$ be the radius of the target orbit (apogee), defined similarly.
    \item $a_t$ be the semi-major axis of the transfer orbit:
    \begin{equation}
        a_t = \frac{r_1 + r_2}{2}.
    \end{equation}
\end{itemize}

\section*{Step-by-Step Process}
\subsection*{1. Velocity Calculations}
The velocity at different points of the orbits is calculated as follows:
\begin{itemize}
    \item Initial orbit velocity:
    \begin{equation}
        v_1 = \sqrt{\frac{\mu}{r_1}}.
    \end{equation}
    \item Velocity at perigee of the transfer orbit:
    \begin{equation}
        v_{t1} = \sqrt{\frac{2\mu}{r_1} - \frac{\mu}{a_t}}.
    \end{equation}
    \item Velocity at apogee of the transfer orbit:
    \begin{equation}
        v_{t2} = \sqrt{\frac{2\mu}{r_2} - \frac{\mu}{a_t}}.
    \end{equation}
    \item Final orbit velocity:
    \begin{equation}
        v_2 = \sqrt{\frac{\mu}{r_2}}.
    \end{equation}
\end{itemize}

\subsection*{2. Delta-v Calculations}
The required impulses (delta-v) for the transfer are given by:
\begin{itemize}
    \item First impulse at perigee of the transfer orbit:
    \begin{equation}
        \Delta v_1 = v_{t1} - v_1.
    \end{equation}
    \item Second impulse at apogee of the transfer orbit:
    \begin{equation}
        \Delta v_2 = v_2 - v_{t2}.
    \end{equation}
\end{itemize}

\subsection*{3. Time of Flight}
The time of flight for the transfer is calculated as half the orbital period of the transfer ellipse:
\begin{equation}
    T_{\text{flight}} = \pi \sqrt{\frac{a_t^3}{\mu}}.
\end{equation}

\subsection*{4. Trajectory Points}
The trajectory points along the transfer orbit are sampled based on the true anomaly $\theta$, which varies from $0$ to $\pi$ during the transfer. The radius at any point is given by:
\begin{equation}
    r(\theta) = \frac{a_t(1 - e^2)}{1 + e \cos(\theta)},
\end{equation}
where $e$ is the eccentricity of the transfer orbit, calculated as:
\begin{equation}
    e = \frac{r_2 - r_1}{r_2 + r_1}.
\end{equation}

The position in Cartesian coordinates can then be calculated as:
\begin{align}
    x &= r(\theta) \cos(\theta), \\
    y &= r(\theta) \sin(\theta).
\end{align}

\section*{Conclusion}
The \texttt{func\_calculate\_transfer} function leverages these equations to compute the delta-v requirements, time of flight, and trajectory points for a Hohmann transfer. The results are encapsulated in a \texttt{Trajectory} object for further use.

\end{document}
